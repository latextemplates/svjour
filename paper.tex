% !TeX spellcheck = en-US
% !TeX encoding = utf8
% !TeX program = pdflatex
% !BIB program = bibtex
% -*- coding:utf-8 mod:LaTeX -*-

% Fix to get cleveref working
% Source: https://tex.stackexchange.com/a/243866/9075
\makeatletter
\def\cl@chapter{}
\makeatother

%\documentclass[natbib,english]{svjour3}                 % onecolumn (standard format)
%\documentclass[smallcondensed,natbib,english]{svjour3}  % onecolumn (ditto)
%\documentclass[smallextended,natbib,english]{svjour3    % onecolumn (second format)
\documentclass[twocolumn,natbib,english]{svjour3}        % twocolumn

% Fix for microtype
% Source: https://tex.stackexchange.com/a/24875/9075
\makeatletter
\if@twocolumn
  \renewcommand\normalsize{%
    \@setfontsize\normalsize\@xpt{12.5pt}%
    \abovedisplayskip=3 mm plus6pt minus 4pt
    \belowdisplayskip=3 mm plus6pt minus 4pt
    \abovedisplayshortskip=0.0 mm plus6pt
    \belowdisplayshortskip=2 mm plus4pt minus 4pt
    \let\@listi\@listI}%

  \renewcommand\small{%
    \@setfontsize\small{8.5pt}\@xpt
    \abovedisplayskip 8.5\p@ \@plus3\p@ \@minus4\p@
    \abovedisplayshortskip \z@ \@plus2\p@
    \belowdisplayshortskip 4\p@ \@plus2\p@ \@minus2\p@
    \def\@listi{\leftmargin\leftmargini
      \parsep 0\p@ \@plus1\p@ \@minus\p@
      \topsep 4\p@ \@plus2\p@ \@minus4\p@
      \itemsep0\p@}%
    \belowdisplayskip \abovedisplayskip}
\else
  \if@smallext
    \renewcommand\normalsize{%
      \@setfontsize\normalsize\@xpt\@xiipt
      \abovedisplayskip=3 mm plus6pt minus 4pt
      \belowdisplayskip=3 mm plus6pt minus 4pt
      \abovedisplayshortskip=0.0 mm plus6pt
      \belowdisplayshortskip=2 mm plus4pt minus 4pt
      \let\@listi\@listI}%

    \renewcommand\small{%
      \@setfontsize\small\@viiipt{9.5pt}%
      \abovedisplayskip 8.5\p@ \@plus3\p@ \@minus4\p@
      \abovedisplayshortskip \z@ \@plus2\p@
      \belowdisplayshortskip 4\p@ \@plus2\p@ \@minus2\p@
      \def\@listi{\leftmargin\leftmargini
        \parsep 0\p@ \@plus1\p@ \@minus\p@
        \topsep 4\p@ \@plus2\p@ \@minus4\p@
        \itemsep0\p@}%
      \belowdisplayskip \abovedisplayskip}
  \else
    \renewcommand\normalsize{%
      \@setfontsize\normalsize{9.5pt}{11.5pt}%
      \abovedisplayskip=3 mm plus6pt minus 4pt
      \belowdisplayskip=3 mm plus6pt minus 4pt
      \abovedisplayshortskip=0.0 mm plus6pt
      \belowdisplayshortskip=2 mm plus4pt minus 4pt
      \let\@listi\@listI}%  

    \renewcommand\small{%
      \@setfontsize\small\@viiipt{9.25pt}%
      \abovedisplayskip 8.5\p@ \@plus3\p@ \@minus4\p@
      \abovedisplayshortskip \z@ \@plus2\p@
      \belowdisplayshortskip 4\p@ \@plus2\p@ \@minus2\p@
      \def\@listi{\leftmargin\leftmargini
        \parsep 0\p@ \@plus1\p@ \@minus\p@
        \topsep 4\p@ \@plus2\p@ \@minus4\p@
        \itemsep0\p@}%
      \belowdisplayskip \abovedisplayskip}
  \fi
\fi
\let\footnotesize\small
\makeatother

% cmap has to be loaded before any font package (such as cfr-lm)
\usepackage{cmap}

%enable margin kerning
% microtype has to be loaded early.
% otherwise, following error appears
%
% Extra \else.
% \XKV@wh@list ...r \expandafter \XKV@wh@list \else
%                                                  \def #3{#6}\expandafter \e...
\RequirePackage[%
  shrink=40,
  final,%
  expansion=alltext,%
  protrusion=alltext-nott]{microtype}%
% \texttt{test -- test} keeps the "--" as "--" (and does not convert it to an en dash)
\DisableLigatures{encoding = T1, family = tt* }

% Set English as language and allow to write hyphenated"=words
%
\usepackage[ngerman,main=english]{babel}
%Hint by http://tex.stackexchange.com/a/321066/9075 -> enable "= as dashes
\addto\extrasenglish{\languageshorthands{ngerman}\useshorthands{"}}

\usepackage{ifluatex}
\ifluatex
  \usepackage{fontspec}
  \usepackage[english]{selnolig}
\fi

\ifluatex
  % use the better (sharper, ...) Latin Modern variant of Computer Modern
  \setmainfont{Latin Modern Roman}
  \setsansfont{Latin Modern Sans}
  \setmonofont{Latin Modern Mono} % "variable=false"
  %\setmonofont{Latin Modern Mono Prop} % "variable=true"
\else
  % better font, similar to the default springer font
  % cfr-lm is preferred over lmodern. Reasoning at http://tex.stackexchange.com/a/247543/9075
  % This obsoletes the backage "fix-cm"
  \usepackage[%
    rm={oldstyle=false,proportional=true},%
    sf={oldstyle=false,proportional=true},%
    tt={oldstyle=false,proportional=true,variable=false},%
    qt=false%
  ]{cfr-lm}
\fi

\ifluatex
\else
  % fontenc and inputenc are not required when using lualatex
  \usepackage[T1]{fontenc}
  \usepackage[utf8]{inputenc} %support umlauts in the input
\fi

\smartqed  % flush right qed marks, e.g. at end of proof

\usepackage{graphicx}

% backticks (`) are rendered as such in verbatim environment. See https://tex.stackexchange.com/a/341057/9075 for details.
\usepackage{upquote}

% Nicer tables (\toprule, \midrule, \bottomrule - see example)
\usepackage{booktabs}

%extended enumerate, such as \begin{compactenum}
\usepackage{paralist}

% For easy quotations: \enquote{text}
% This package is very smart when nesting is applied, otherwise textcmds (see below) provides a shorter command
\usepackage{csquotes}

% For even easier quotations: \qq{text}
\usepackage{textcmds}

%tweak \url{...}
\usepackage{url}
%\urlstyle{same}
%improve wrapping of URLs - hint by http://tex.stackexchange.com/a/10419/9075
\makeatletter
\g@addto@macro{\UrlBreaks}{\UrlOrds}
\makeatother
%nicer // - solution by http://tex.stackexchange.com/a/98470/9075
%DO NOT ACTIVATE -> prevents line breaks
%\makeatletter
%\def\Url@twoslashes{\mathchar`\/\@ifnextchar/{\kern-.2em}{}}
%\g@addto@macro\UrlSpecials{\do\/{\Url@twoslashes}}
%\makeatother

% Diagonal lines in a table - http://tex.stackexchange.com/questions/17745/diagonal-lines-in-table-cell
% Slashbox is not available in texlive (due to licensing) and also gives bad results. This, we use diagbox
%\usepackage{diagbox}

\usepackage[mode=text,group-four-digits,group-separator={,}]{siunitx}
\sisetup{locale=US}
\newcommand{\numt}[1]{\num{#1}}

% Required for package pdfcomment later
\usepackage{xcolor}

% For listings
\usepackage{listings}
\lstset{%
  basicstyle=\ttfamily,%
  columns=fixed,%
  basewidth=.5em,%
  xleftmargin=0.5cm,%
  captionpos=b}%
\renewcommand{\lstlistingname}{List.}
% Fix counter as described at https://tex.stackexchange.com/a/28334/9075
\usepackage{chngcntr}
\AtBeginDocument{\counterwithout{lstlisting}{section}}

% Enable nice comments
\usepackage{pdfcomment}
%
\newcommand{\commentontext}[2]{\colorbox{yellow!60}{#1}\pdfcomment[color={0.234 0.867 0.211},hoffset=-6pt,voffset=10pt,opacity=0.5]{#2}}
\newcommand{\commentatside}[1]{\pdfcomment[color={0.045 0.278 0.643},icon=Note]{#1}}
%
% Compatibality with packages todo, easy-todo, todonotes
\newcommand{\todo}[1]{\commentatside{#1}}
% Compatiblity with package fixmetodonotes
\newcommand{\TODO}[1]{\commentatside{#1}}

\newcommand*{\doi}[1]{\href{https://doi.org/\detokenize{#1}}{DOI: \detokenize{#1}}}

% Put footnotes below floats
% Source: https://tex.stackexchange.com/a/32993/9075
\usepackage{stfloats}
\fnbelowfloat

% Enable correct jumping to figures when referencing
\usepackage[all]{hypcap}

\usepackage[acronym,indexonlyfirst,nomain]{glossaries}
\glsdisablehyper
\input{abbrevations}

%enable \cref{...} and \Cref{...} instead of \ref: Type of reference included in the link
\usepackage[capitalise,nameinlink]{cleveref}
%Nice formats for \cref
\crefname{section}{Sect.}{Sect.}
\Crefname{section}{Section}{Sections}
\crefname{listing}{\lstlistingname}{\lstlistingname}
\Crefname{listing}{Listing}{Listings}

%Intermediate solution for hyperlinked refs. See https://tex.stackexchange.com/q/132420/9075 for more information.
\newcommand{\Vlabel}[1]{\label[line]{#1}\hypertarget{#1}{}}
\newcommand{\lref}[1]{\hyperlink{#1}{\FancyVerbLineautorefname~\ref*{#1}}}

\usepackage{xspace}
%\newcommand{\eg}{e.\,g.\xspace}
%\newcommand{\ie}{i.\,e.\xspace}
\newcommand{\eg}{e.\,g.,\ }
\newcommand{\ie}{i.\,e.,\ }

%introduce \powerset - hint by http://matheplanet.com/matheplanet/nuke/html/viewtopic.php?topic=136492&post_id=997377
\DeclareFontFamily{U}{MnSymbolC}{}
\DeclareSymbolFont{MnSyC}{U}{MnSymbolC}{m}{n}
\DeclareFontShape{U}{MnSymbolC}{m}{n}{
  <-6>    MnSymbolC5
  <6-7>   MnSymbolC6
  <7-8>   MnSymbolC7
  <8-9>   MnSymbolC8
  <9-10>  MnSymbolC9
  <10-12> MnSymbolC10
  <12->   MnSymbolC12%
}{}
\DeclareMathSymbol{\powerset}{\mathord}{MnSyC}{180}


% correct e.g. and e.g.,
\renewcommand{\eg}{e.\,g.,\xspace}
\newcommand{\egc}{e.\,g.,\xspace}


% correct bad hyphenation here
\hyphenation{op-tical net-works semi-conduc-tor data-bases Berm-bach}

% Enable hyperref without colors and without bookmarks
\hypersetup{hidelinks,
  colorlinks=true,
  allcolors=black,
  pdfstartview=Fit,
  breaklinks=true}

\journalname{JOURNALNAME}

% For demonstration purposes only
\usepackage[math]{blindtext}
\usepackage{mwe}

\begin{document}

\title{TITLE}
%\subtitle{Do you have a subtitle?\\ If so, write it here}

%\titlerunning{Short form of title}        % if too long for running head

\author{First Author \and Second Author}

%\authorrunning{Short form of author list} % if too long for running head

\institute{
  F.~Author\at
  University of Something \\
  \email{fauthor@example.org}
  \and
  S.~Author\at
  University of Anything \\
  \email{sauthor@example.org}
}


\date{Received: date / Accepted: date}
% The correct dates will be entered by the editor

\maketitle

\begin{abstract}
  \lipsum[1]

  \keywords{keyword1 \and keyword2}
\end{abstract}

\section{Introduction}
\label{sec:intro}
\lipsum[1-3]\todo{Refine me}

The remainder of the paper starts with a presentation of related work (\cref{sec:relatedwork}).
It is followed by a presentation of hints on \LaTeX{} (\cref{sec:hints}).
Finally, a conclusion is drawn and outlook on future work is made (\cref{sec:outlook}).

\section{Related Work}
\label{sec:relatedwork}

Winery~\citep{Winery} is a graphical \commentontext{modeling}{modeling with one \qq{l}, because of AE} tool.
The whole idea of TOSCA is explained by \citet{Binz2009}.

\section{LaTeX Hints}
\label{sec:hints}

\begin{figure}
  \centering
  \includegraphics[width=.8\linewidth]{example-image-golden}
  \caption{Simple Figure \citep[based on][]{mwe}.}
  \label{fig:simple}
\end{figure}

\begin{table}
  \caption{Simple Table}
  \label{tab:simple}
  \centering
  \begin{tabular}{ll}
    \toprule
    Heading1 & Heading2 \\
    \midrule
    One      & Two      \\
    Thee     & Four     \\
    \bottomrule
  \end{tabular}
\end{table}

\begin{lstlisting}[
  % one can adjust spacing here if required
  % aboveskip=2.5\baselineskip,
  % belowskip=-.8\baselineskip,
  caption={Example Java Listing},
  label=L1,
  language=Java,
  float]
public class Hello {
    public static void main (String[] args) {
        System.out.println("Hello World!");
    }
}
\end{lstlisting}

\begin{lstlisting}[
  % one can adjust spacing here if required
  % aboveskip=2.5\baselineskip,
  % belowskip=-.8\baselineskip,
  caption={Example XML Listing},
  label=L2,
  language=XML,
  float]
<example attr="demo">
  text content
</example>
\end{lstlisting}

\Cref{L1,L2} show listings typeset using the \texttt{lstlisting} environment.

\href{https://ctan.org/pkg/cleveref}{Cleveref -- Intelligent cross-referencing} demonstration: \texttt{Cref} at beginning of sentence, \texttt{cref} in all other cases.
Look at the \LaTeX{} source for the effects.
\Cref{fig:simple} shows a simple fact, although \cref{fig:simple} could also show something else.
\Cref{tab:simple} shows a simple fact, although \cref{tab:simple} could also show something else.
\Cref{sec:intro} shows a simple fact, although \cref{sec:intro} could also show something else.

Brackets work as designed:
<test>
One can also input backquotes in verbatim text: \verb|`test`|.

The symbol for powerset is now correct: $\powerset$ and not a Weierstrass p ($\wp$).

\begin{inparaenum}
  \item All these items...
  \item ...appear in one line
  \item This is enabled by the paralist package.
\end{inparaenum}

Please use the \qq{qq command} or the \enquote{enquote command} to quote something.
``something in quotes'' using plain tex syntax also works.

You can now write words containing hyphens which are hyphenated (application"=specific) at other places.
This is enabled by an additional configuration of the babel package.
In case you write \qq{application-specific}, then the word will only be hyphenated at the dash.
You can also write applica\allowbreak{}tion-specific, but this is much more effort.

The words \qq{workflow} and \qq{dwarflike} can be copied from the PDF and pasted to a text file.

Numbers can written plain text (such as 100), by using the siunitx package like that:
\SI{100}{\km\per\hour},
or by using plain \LaTeX{} (and math mode):
$100 \frac{\mathit{km}}{h}$.

\section{Conclusion and Outlook}
\label{sec:outlook}
\lipsum[1-2]

\begin{acknowledgements}
  \lipsum[1].
\end{acknowledgements}

In the bibliography, use \texttt{\textbackslash textsuperscript} for \qq{st}, \qq{nd}, \ldots:
E.g., \qq{The 2\textsuperscript{nd} conference on examples}.
When you use \href{https://www.jabref.org}{JabRef}, you can use the clean up command to achieve that.
See \url{https://help.jabref.org/en/CleanupEntries} for an overview of the cleanup functionality.

% spbasic: Springer medical, life sciences, chemistry, geology, engineering and computer science publications
% spmpsci.bst: Springer mathematics, computer science, and physical sciences journals publications.   
% spphys.bst: Springer physics publications
\bibliographystyle{spbasic}
\bibliography{paper}

\phantom{.}

All links were last followed on October 5, 2019.
\end{document}
